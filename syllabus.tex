\documentclass[12pt,]{article}
\usepackage{lmodern}
\usepackage{amssymb,amsmath}
\usepackage{ifxetex,ifluatex}
\usepackage{fixltx2e} % provides \textsubscript
\ifnum 0\ifxetex 1\fi\ifluatex 1\fi=0 % if pdftex
  \usepackage[T1]{fontenc}
  \usepackage[utf8]{inputenc}
\else % if luatex or xelatex
  \ifxetex
    \usepackage{mathspec}
  \else
    \usepackage{fontspec}
  \fi
  \defaultfontfeatures{Ligatures=TeX,Scale=MatchLowercase}
\fi
% use upquote if available, for straight quotes in verbatim environments
\IfFileExists{upquote.sty}{\usepackage{upquote}}{}
% use microtype if available
\IfFileExists{microtype.sty}{%
\usepackage{microtype}
\UseMicrotypeSet[protrusion]{basicmath} % disable protrusion for tt fonts
}{}
\usepackage[margin=1in]{geometry}
\usepackage{hyperref}
\hypersetup{unicode=true,
            pdfborder={0 0 0},
            breaklinks=true}
\urlstyle{same}  % don't use monospace font for urls
\usepackage{graphicx,grffile}
\makeatletter
\def\maxwidth{\ifdim\Gin@nat@width>\linewidth\linewidth\else\Gin@nat@width\fi}
\def\maxheight{\ifdim\Gin@nat@height>\textheight\textheight\else\Gin@nat@height\fi}
\makeatother
% Scale images if necessary, so that they will not overflow the page
% margins by default, and it is still possible to overwrite the defaults
% using explicit options in \includegraphics[width, height, ...]{}
\setkeys{Gin}{width=\maxwidth,height=\maxheight,keepaspectratio}
\IfFileExists{parskip.sty}{%
\usepackage{parskip}
}{% else
\setlength{\parindent}{0pt}
\setlength{\parskip}{6pt plus 2pt minus 1pt}
}
\setlength{\emergencystretch}{3em}  % prevent overfull lines
\providecommand{\tightlist}{%
  \setlength{\itemsep}{0pt}\setlength{\parskip}{0pt}}
\setcounter{secnumdepth}{0}
% Redefines (sub)paragraphs to behave more like sections
\ifx\paragraph\undefined\else
\let\oldparagraph\paragraph
\renewcommand{\paragraph}[1]{\oldparagraph{#1}\mbox{}}
\fi
\ifx\subparagraph\undefined\else
\let\oldsubparagraph\subparagraph
\renewcommand{\subparagraph}[1]{\oldsubparagraph{#1}\mbox{}}
\fi

%%% Use protect on footnotes to avoid problems with footnotes in titles
\let\rmarkdownfootnote\footnote%
\def\footnote{\protect\rmarkdownfootnote}

%%% Change title format to be more compact
\usepackage{titling}

% Create subtitle command for use in maketitle
\newcommand{\subtitle}[1]{
  \posttitle{
    \begin{center}\large#1\end{center}
    }
}

\setlength{\droptitle}{-2em}
  \title{}
  \pretitle{\vspace{\droptitle}}
  \posttitle{}
  \author{}
  \preauthor{}\postauthor{}
  \date{}
  \predate{}\postdate{}

\usepackage{multirow}

\begin{document}

\section{Foundations of Quantitative Reasoning
(BIO381)}\label{foundations-of-quantitative-reasoning-bio381}

\begin{center}\rule{0.5\linewidth}{\linethickness}\end{center}

\section{Syllabus and Course Outline}\label{syllabus-and-course-outline}

University of Vermont

Spring 2019

Instructors: \textbf{Dr.~Easton R. White}

email:
\href{mailto:Easton.White@uvm.edu}{\nolinkurl{Easton.White@uvm.edu}}

Time: M/W 3:00pm - 5:00pm + additional lab time on Friday?

Location: Hills 226

\section{Course Introduction}\label{course-introduction}

Welcome to the \href{https://www.uvm.edu/quest}{QuEST} Foundations of
Quantitative Reasoning course. This course is designed to provide
students with a solid background in both ecology and evolution as well
as commonly used quantitative tools. By the end of the course, students
should be able to:

\subsubsection{Concepts}\label{concepts}

\begin{enumerate}
\def\labelenumi{\arabic{enumi}.}
\tightlist
\item
  Understand basic concepts in population and community ecology
\item
  Compare and contrast different modeling philosophies
\item
  Analyze simple population models using qualitative approaches
\item
  Use linear algebra to construct and analyze structured population
  models
\item
  Understand foundations of spatial ecology and build simple models
\item
  Compare and contrast hypotheses that attempt to explain basic patterns
  in biodiversity.
\item
  Understand principles of evolution and natural selection
\item
  Understand and apply various tools to understand evolution (population
  genetics, quantitative genetics, and game theory)
\item
  Understand and apply simple models to study infectious disease
  dynamics
\item
  Describe how humans affect ecological and evolutionary dynamics
\item
  Read and discuss papers that involve modeling
\item
  Construct and program simple population models in R
\item
  Understand and apply a variety of mathematical tools (equilibrium and
  stability analysis, linear algebra, stochastic models, etc.)
\item
  Work in small, collaborative groups to find solutions to particular
  case studies.
\end{enumerate}

\subsubsection{Skills}\label{skills}

This is clearly a lot of objectives and material to cover in one
semester. However, there will be lots of time to ask questions and to
have discussions. The format of the class will change from day to day,
but will include a combination of lecturing, problem-solving activities,
discussions, and working in small groups.

As a four-credit course, you should expect do spend at least six hours a
week reading and working on problems. The more time and effort you put
in, the more you will get out of this course.

Any and all of the content of this syllabus is subject to change as we
go through the course. The material will be tailored to fit the needs of
the class. Changes will be announced in-class and electronically.

\section{Course resources}\label{course-resources}

\paragraph{Your classmates}\label{your-classmates}

By its very nature, this course will challenge each student during
different points. The ecology and evolution may be challenging for those
with little biology background and the quantitative skills may be
difficult to pick up for those of you with traditional biology
backgrounds. It will be essential to work with others on
assignments---just be sure that you understand the material yourself.

\paragraph{Your instuctor}\label{your-instuctor}

I will hold office hours several times per week and I am always happy to
set up additional times to meet. Please ask for help, early and often.
If you have a question, others in the class surely have the same one. At
any point, please reach out to let me know how I might improve the
course.

\paragraph{R office hours}\label{r-office-hours}

In addition to normal office hours, every other week there will be an R
work time in Jeffords. This is a time to get help on R for this class,
other classes, or for your research. Occasionally, mini-workshops on
various R topics will also be taught during this time.

\paragraph{Various textbooks}\label{various-textbooks}

This course will not follow any particular textbook. However, I will
draw from several books and online sources:

\begin{itemize}
\tightlist
\item
  \url{https://github.com/cooplab/popgen-notes}
\item
  \url{https://www.amazon.com/Primer-Ecology-Fourth-Nicholas-Gotelli/dp/0878933182/ref=pd_bbs_sr_1?ie=UTF8\&s=books\&qid=1231604546\&sr=8-1}
\item
  \url{https://www.amazon.com/Population-Biology-Concepts-Alan-Hastings/dp/0387948538}
\end{itemize}

\paragraph{Various online tools}\label{various-online-tools}

Calculus a bit rusty? Maybe you're not totally remembering Mendelian
genetics? Something else? We will walk through topics from the
beginning, but there might be points that still seem confusing. Besides
your classmates and instructor, there are a plethora of online tools to
help provide a quick refresher:

\begin{itemize}
\tightlist
\item
  \url{http://tutorial.math.lamar.edu/}
\item
  \url{https://www.khanacademy.org/}
\end{itemize}

\section{Grading Standards and
Practices}\label{grading-standards-and-practices}

\paragraph{Problem sets: 12 @ 5 points
each}\label{problem-sets-12-5-points-each}

Every week there will be assigned problem sets. These will involve a mix
of math done on paper and programming in R. The problem sets will either
be turned in on paper or on Github depending on the assignment. Some
problem sets will be individual-based and others will be in small teams.

\paragraph{Group project: 30 points}\label{group-project-30-points}

The final project will involve building a model to understand some
system. The choice of system, modeling approach, and question will be up
to each team. The teams will be between 2-3 students. Think of this
project as an opportunity that could lead to a publication.

\paragraph{Teaching a lesson: 30
points}\label{teaching-a-lesson-30-points}

Everyone will be expected to teach one class (or lab) during the course
of the semester. The specific lesson will be chosen by the student. The
student will work with the instructor to design the lesson in line with
learning outcomes for the course.

\paragraph{Participation: 30 points}\label{participation-30-points}

Students will be expected to be engaged\ldots{}

\section{Academic honesty}\label{academic-honesty}

Students in the class are expected to complete their own work. It is
encouraged that you work with other students outside of class, but your
completed work must be from you and in your own words. Students are
expected to follow UVM policy on plagiarism, fabrication, collusion, and
cheating: \url{http://www.uvm.edu/policies/student/acadintegrity.pdf}.

\section{Student Learning
Accomadations}\label{student-learning-accomadations}

In keeping with University policy, any student with a documented
disability interested in utilizing accommodations should contact SAS,
the office of Disability Services on campus. SAS works with students and
faculty in an interactive process to explore reasonable and appropriate
accommodations, which are communicated to faculty in an accommodation
letter. All students are strongly encouraged to meet with their faculty
to discuss the accommodations they plan to use in each course. A
student's accommodation letter lists those accommodations that will not
be implemented until the student meets with their faculty to create a
plan. Contact SAS: A170 Living/Learning Center; 802-656-7753.

Contact Student Accessibility Services (SAS):

\begin{itemize}
\tightlist
\item
  A170 Living/Learning Center
\item
  802-656-7753
\item
  \href{mailto:access@uvm.edu}{\nolinkurl{access@uvm.edu}}
\item
  \url{https://www.uvm.edu/academicsuccess/student_accessibility_services}
\end{itemize}

\section{Student health and
wellbeing}\label{student-health-and-wellbeing}

There are several resources on campus dedicated to promoting student
health and well-being, including:

\begin{itemize}
\item
  Center for Health and Wellbeing \url{https://www.uvm.edu/health}
\item
  Counseling \& Psychiatry Services (CAPS)
  \url{https://www.uvm.edu/health/CAPS}
\end{itemize}


\end{document}
